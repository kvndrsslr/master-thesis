%*******************************************************
% Abstract
%*******************************************************
%\renewcommand{\abstractname}{Abstract}
\pdfbookmark[1]{Abstract}{Abstract}
% \addcontentsline{toc}{chapter}{\tocEntry{Abstract}}
\begingroup
\let\clearpage\relax
\let\cleardoublepage\relax
\let\cleardoublepage\relax

\chapter*{Abstract}
The impact of Linked Data in applications on the World Wide Web as well as in businesses has been tremendous in the recent decade.
Linked Data Integration is of paramount importance for this tendency to continue.
Despite there being a great amount of literatue on lifting, interlinking and fusion of RDF Datasets, the field of RDF dataset enrichment has seen little attention.
Specifically, there exist no accessible tools to power this step in the Linked Data Lifecycle.

This thesis attempts to address this gap by extending on a previous approach in RDF dataset enrichment to enable successful RDF dataset enrichment for a larger set of use cases.
Since such tools require extensive configuration of their components in order to lead to satisfactory results, it is often not feasible for non experts to use them.
Therefore, an efficient machine learning algorithm using novel improvements to enable the use of our tool by novice user is proposed.
Evaluation suggests that our approach can successfully learn a larger class of RDF dataset enrichment specifications than the state of the art, using only a single training example.

\newpage

\begin{otherlanguage}{ngerman}
\pdfbookmark[1]{Zusammenfassung}{Zusammenfassung}
\chapter*{Zusammenfassung}
Linked Data konnten in der letzten Dekade in Anwendungen des World Wide Webs und der Wirtschaft große Erfolge erringen.
Die Integration von Linked Data ist ein essentieller Bestandteil um diesen Erfolg fortzuführen.
Obwohl das Liften, Verlinken und Fusionieren von RDF-Datensätzen in großem Umfang in der Literatur behandelt wird, hat das Gebiet der RDF-Datensatz-anreicherung wenig Beachtung gefunden.
Insbesondere existieren keine zugänglichen Werkzeuge, um diesen Schritt im Linked Data Lifecycle zu unterstützen.

Diese Arbeit versucht, eben jene Lücke zu schließen, indem sie einen bestehenden Ansatz der RDF-Datensatzanreicherung mit dem Ziel erweitert, die RDF-Datensatzanreicherung in einer größeren Klasse von Anwendungsfällen zu ermöglichen.
Da solche Werkzeuge eine umfangreiche Konfiguration ihrer Komponenten erfordern, um zu zufriedenstellenden Resultaten zu führen, ist es für Laien häufig zu schwer, diese zu verwenden.
Daher wird auch ein effizienter Algorithmus für maschinelles Lernen entwickelt, der neuartige Verbesserungen verwendet, um die Verwendung unseres Werkzeuges für unerfahrene Benutzer zu ermöglichen.
Die Evaluation legt nahe, dass dieser Ansatz in der Lage ist, eine größere Klasse von RDF-Datensatzanreicher-ungsspezifikationen erfolgreich zu erlernen als bestehende vergleichbare Ansätze.
Zudem benötigt er dafür nur ein einziges Trainingsbeispiel.
\end{otherlanguage}

\endgroup

\vfill
